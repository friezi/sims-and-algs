\documentclass[a4paper]{article}
\usepackage[utf8]{inputenc}
\usepackage{latexsym}
\usepackage[english]{babel}
\usepackage[a4paper]{geometry}
\usepackage[sumlimits]{amsmath}
\usepackage{amssymb}
\usepackage[nomessages]{fp}
\usepackage{pst-all}
\usepackage[labelformat=simple]{caption}
\usepackage{multido}

\parindent0pt

\title{Calculation of momentum velocity on collision of two masses}
\author{Friedemann Zintel}
\date{\today}

\newcommand{\spc}{\hspace{0.2cm}}

\begin{document}
\maketitle

In order to calculate the momentum between two colliding masses it is necessary to determine the partial velocity vector of a moving mass in direction to the second mass.
Let's assume a mass $m_1$ moves with velocity $\vec{v}$ and would hit mass $m_2$. For simplicity $m_2$ remains stationary and the shapes of both masses are spherical.
Even when the direction of $\vec{v}$ does not point directly to $m_2$, it behaves as if $m_1$ hits $m_2$ with a (partial) velocity $\vec{v_{m_2}}$ which points in the direction to $m_2$. We keep the calculation in two dimensional space.
It should be easy to extend it to three dimensions.

\begin{center}
\psset{xunit=1pt,yunit=1pt,runit=1pt}
\begin{pspicture}(0,150)(600,350)
  \newcommand{\msx}{150}
  \newcommand{\msy}{254}
  \newcommand{\bradius}{30}
  \newcommand{\cradius}{2}
  \newcommand{\mtx}{199}
  \newcommand{\mty}{220}
  \newcommand{\vlen}{273}
  \FPeval{\rsy}{clip(\msy+\bradius)}
  \FPeval{\rty}{clip(\mty+\bradius)}
  \FPeval{\lrsy}{clip(\msy+\bradius/2)}
  \FPeval{\lrty}{clip(\mty+\bradius/2)}
  \FPeval{\mvlsx}{clip(\msx-75)}
  \FPeval{\mvlsy}{clip(\msy)}
  \FPeval{\mvltx}{clip(\mtx+140)}
  \FPeval{\mvlty}{clip(\msy)}
  \FPeval{\mmlsx}{clip(\msx-75)}
  \FPeval{\mmlsy}{clip(\msy+51)}
  \FPeval{\mmltx}{clip(\mtx+75)}
  \FPeval{\mmlty}{clip(\mty-51)}
  \FPeval{\vx}{clip(\msx+(\vlen-\msx)/2)}
  \FPeval{\vmsy}{clip(\msy+3)}
  \FPeval{\vmty}{clip(\mty-25+3)}
  \FPeval{\vmtx}{clip(\mtx+35+3)}
  \FPeval{\mxra}{clip(\vmtx-2)}
  \FPeval{\myra}{clip(\vmty+7)}
  \FPeval{\dx}{clip(\msx+(\mtx-\msx)/2)}
  \FPeval{\dy}{clip(\mty+(\msy-\mty)/2)}
  \FPeval{\sx}{clip(\vmtx+(\vlen-\vmtx)/2)}
  \FPeval{\sy}{clip(\vmty+(\msy-\vmty)/2)}

  % m1
  \pscircle(\msx,\msy){\bradius}
  \pscircle*(\msx,\msy){\cradius}
  \uput[dl](\msx,\msy){$m_1$}
  \psline[]{-}(\msx,\msy)(\msx,\rsy)
  \uput[r](\msx,\lrsy){$r_1$}

  %m2
  \pscircle(\mtx,\mty){\bradius}
  \pscircle*(\mtx,\mty){\cradius}
  \uput[dl](\mtx,\mty){$m_2$}
  \psline[]{-}(\mtx,\mty)(\mtx,\rty)
  \uput[r](\mtx,\lrty){$r_2$}

  % movement line
  \psline[linestyle=dashed]{-}(\mvlsx,\mvlsy)(\mvltx,\mvlty)
  \uput[u](\mvltx,\mvlty){movement line}

  % vector v
  \psline[arrowscale=2]{->}(\msx,\msy)(\vlen,\msy)
  \uput[u](\vx,\msy){$\vec{v}$}

  % momentum line
  \psline[linestyle=dashed]{-}(\mmlsx,\mmlsy)(\mmltx,\mmlty)
  \uput[ur](\mmltx,\mmlty){momentum line}

  % vector d
  \psline[arrowscale=2,linecolor=blue]{->}(\msx,\msy)(\mtx,\mty)
  \uput[d](\dx,\dy){\textcolor{blue}{$\vec{d}$}}

  % vector vm2
  \psline[arrowscale=2,linecolor=red]{->}(\msx,\vmsy)(\vmtx,\vmty)
  \uput[dl](\vmtx,\vmty){\textcolor{red}{$\vec{v_{m_2}}$}}

  % vector s
  \psline[arrowscale=2,linecolor=olive]{->}(\vlen,\msy)(\vmtx,\vmty)
  \uput[dr](\sx,\sy){\textcolor{olive}{$\vec{s}$}}

  % right angle
  \psarc{-}(\vmtx,\vmty){15}{60}{140}
  \pscircle*(\mxra,\myra){1}

\end{pspicture}
\captionof{figure}{collision}
\label{fig:coll}
\end{center}

We need to calculate $\vec{v_{m_2}}$. We define $\vec{d}=\vec{m_2}-\vec{m_1}$, where $\vec{m_1}$ and $\vec{m_2}$ are the positional vectors for $m_1$ resp. $m_2$.
We require that the masses have a positive expansion ($r_1>0$, $r_2>0$), thus $\vec{d} \neq \vec{0}$ and $d_x \neq 0 \vee d_y \neq 0$ (collision takes place if
$|\vec{d}|=r_1+r_2$). Let $\vec{s}$ be a (the) vector with
$\vec{v_{m_2}}=\vec{v}+\vec{s}$. Then $\vec{s}$ must be orthogonal to $\vec{v_{m_2}}$ and thus to $\vec{d}$. That is because $\vec{v_{m_2}}$ is a partial vector of $\vec{v}$
and points in the same direction as $\vec{d}$ which resides on the momentum line. See figure \ref{fig:coll}.\\
Be $\lambda \in \mathbb{R}$, then the following equations hold:

\begin{displaymath}
  \vec{v_{m_2}}=\vec{v}+\vec{s}=\lambda \vec{d}
\end{displaymath}
\begin{displaymath}
  \vec{s}\cdot\vec{d}=0
\end{displaymath}

We end up with three equations and three unkown variables ($s_x,s_y,\lambda$), which can easily be solved:

\begin{equation}
  v_x+s_x=\lambda d_x
  \label{eq:vx}
\end{equation}
\begin{equation}
  v_y+s_y=\lambda d_y
  \label{eq:vy}
\end{equation}
\begin{equation}
  s_xd_x+s_yd_y=0
  \label{eq:nvec}
\end{equation}

case: \boxed{d_x \neq 0}
\begin{align}
  (\ref{eq:nvec}) <=> s_x=-\frac{s_y d_y}{d_x}\\
  (\ref{eq:vy}) <=> s_y=\lambda d_y-v_y\\
  => s_x=-\frac{(\lambda d_y-v_y)d_y}{d_x}\\
  in \spc (\ref{eq:vx}) : v_x-\frac{(\lambda d_y-v_y)d_y}{d_x}=\lambda d_x\\
  <=> v_x-\frac{\lambda d_y^2}{d_x}+\frac{v_yd_y}{d_x}=\lambda d_x\\
  <=> v_x+\frac{v_yd_y}{d_x}=\lambda (d_x+\frac{d_y^2}{d_x})\\
  <=>\frac{v_xd_x}{d_x}+\frac{v_yd_y}{d_x}=\lambda (\frac{d_x^2+d_y^2}{d_x})\\
  <=>\frac{(v_xd_x+v_yd_y)d_x}{d_x(d_x^2+d_y^2)}=\lambda\\
  <=>\lambda=\frac{v_xd_x+v_yd_y}{d_x^2+d_y^2}=\frac{\vec{v}\vec{d}}{\vec{d}^2}
\end{align}

case: \boxed{d_y \neq 0}\\\\
Calculation is analogue. We end up with the same result for $\lambda$.\\\\

As a result we get:

\begin{displaymath}
  \vec{v_{m_2}}=\frac{\vec{v}\vec{d}}{\vec{d}^2}\vec{d}
\end{displaymath}
\\
$\square$
\end{document}